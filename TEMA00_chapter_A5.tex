La confección de este texto es fruto de una larga experiencia como profesor de matemáticas de secundaria y para ello me he basado en mis más de treinta años de docencia  y en la de tantos autores que han contribuido a la explicación de estos conceptos a multitud de alumnos. En especial me he basado en los apuntes de `Cálculo diferencial e integral de funciones de una variable', del profesor Francisco Javier Pérez González, del Departamento de Análisis Matemático de la Universidad de Granada y apuntes del profesor J.A. Gálvez, del departamento de geometría y topología también de la universidad de Granada (ambos apuntes los encontré en la red), así como textos célebres como el Thomson, Apostol, Spivak y otros. También apuntes y problemas de libros de texto de segundo de bachillerato como el que usamos en mi IES los últimos años Anaya, asímismo he acudido a  los de la `Marea Verde', SM, `Curso práctico de matemáticas de COU' de ed. Edunsa y otros;  así como los apuntes y problemas encontrados en la web y, en ocasiones, de la inestimable Wikipedia.También he usado exámenes de EBAU de varias comunidades autónomas. Gracias a todos ellos por su inestimable ayuda para la confección de este pequeño texto que espero que sirva a alguien y que escribo libre de todo tipo de derechos.

Los apartados, teoremas y los ejercicios de mayor nivel estarán marcados con el símbolo $\divideontimes$. Exceden los contenidos de un curso de segundo de bachillerato pero son altamente recomendables para el alumnado que necesite ampliar sus conocimientos matemáticos en cursos posteriores.

\section{Cómo estudiar matemáticas}

 
Las Matemáticas son una asignatura que no deja indiferente a ningún estudiante. Algunos la aman y otros la odian; siendo este segundo grupo mucho más numeroso que el primero en la mayoría de las ocasiones. Sin embargo, muchos de los estudiantes que odian las matemáticas lo hacen porque no saben cómo estudiarlas para obtener buenos resultados.

Las Matemáticas son una de esas asignaturas en las que las horas de estudio no tienen una relación directa con la nota. Por mucho que hayas estudiado, si no eres capaz de solucionar el problema del examen, estás perdido. No obstante, existen algunas técnicas para aprender matemáticas que pueden hacer que, independientemente de tu nivel, le saques más partido a tu tiempo de estudio y aumentes tus probabilidades de éxito. ¡Hasta es posible que te acabes uniendo al grupo de amantes de las matemáticas!

\emph{Cómo Estudiar Matemáticas:}
 
\begin{enumerate}
	 

\item Práctica, Práctica y Más Práctica

Es imposible aprender matemáticas leyendo y escuchando. Para aprender matemáticas hay que ponerse el mono de trabajo y lanzarse a hacer ejercicios matemáticos. Cuanto más practiques, mejor. Cada ejercicio tiene sus particularidades y es importante haber realizado el máximo número de ejercicios posibles antes de enfrentarnos al examen. Este punto es el más importante de todos y la base del resto de técnicas para estudiar matemáticas de esta lista.

Una vez que entendiste los ejemplos explicados en clase, desarrolla ejercicios o problemas con solución para complementar tus conocimientos. Trata de hacerlos sin ver la respuesta y luego, si ni lo puedes terminar, ayúdate de la solución y sigue desarrollando, siempre preguntándote ?`por qué se realiza dicho paso? Se trata de que poco a poco no tengas necesidad de utilizar las soluciones y llegará el momento, después de unos pocos ejercicios, en que ya no necesites de ellas para desarrollar y entender completamente los problemas

No pienses que escuchando la explicación de muchos ejercicios y problemas, donde un profesor te explica y entiendes un 100$\%$, es lo más importante para aprender Matemáticas. La clave del éxito es \emph{¡PRACTICAR!}

\item Revisa los Errores

Cuando estés practicando con ejercicios, es muy importante que compruebes los resultados y, más importante aún, que te detengas en la parte que has fallado y examines el proceso en detalle hasta asimilarlo. De nada sirve comparar resultados si no sabes en qué te has equivocado. Por eso es conveniente que tengas unos buenos apuntes con problemas resueltos. De esta manera, evitarás cometer los mismos fallos en el futuro. También es recomendable apuntar todos tus fallos y repasarlos repetidamente antes del examen.

Toma buenos apuntes: Sé ordenado, utiliza un solo cuaderno para el curso y escribe claramente usando, en lo posible, tus propias palabras para que puedas entender cuando estudies. Copia todo lo que el profesor diga y escriba en la pizarra, anotando los porqués de cada paso, ya que uno siempre puede olvidar lo escuchado y cuando vuelvas a leer tus apuntes podrás recordarlos rápidamente.

Observa y apunta si el profesor hace hincapié en ciertos puntos basándose en repeticiones, ejemplos, diagramas, comentarios extensos, etc., éstos son casi siempre parte importante de los temas.

\item Domina los Conceptos Clave

¡No intentes aprenderte los problemas de memoria! Los problemas matemáticos pueden tener miles de variantes y particularidades, por lo que es inútil aprendernos problemas de memoria sin entenderlos. Es cambio, es mucho más efectivo dominar los conceptos importantes y el proceso de resolución de los problemas.

Recuerda que las Matemáticas son una asignatura secuencial, por lo que es importante asentar una base firme dominando los conceptos clave y teniendo claras las fórmulas matemáticas esenciales.

\item Consulta tus Dudas

Puede que en muchas ocasiones te sientas atascado en una parte de un problema o que simplemente no entiendas el proceso. Lo común en estos casos es simplemente pasar de ese problema y pasar al siguiente. Sin embargo, es recomendable despejar todas las dudas que tengas en la resolución de un problema.

Por tanto, puede ser buena idea estudiar junto a algún/a compañero/a con el que consultar dudas y trabajar juntos en problemas más complejos. O, mejor todavía, ¿por qué no te unes a un grupo de estudio en el que puedes plantear tus dudas y trabajar colaborativamente? Asimismo, recuerda plantearle al profesor/a las dudas que tengas, ya sea en clase o en una tutoría.

\item Crea un Ambiente de Estudio sin Distracciones.

Las Matemáticas son una asignatura que requiere más concentración que ninguna otra. Un ambiente de estudio adecuado y libre de distracciones puede ser el factor determinante para conseguir resolver ecuaciones o problemas de geometría, álgebra, trigonometría o complejos. Si te gusta estudiar con música, puede ser una buena idea escucharla de fondo para relajarte y favorecer un ambiente de máxima concentración; la música instrumental es la más recomendable en estas ocasiones.

¡Ah, y no olvides que es importante también tener confianza en uno mismo y afrontar el examen sabiendo que te has preparado adecuadamente!

\end{enumerate}

\emph{Empieza a Estudiar Matemáticas Ahora. ¡Es gratis!}

\section{Guía de lectura}	

En el apéndice \ref{Kit} tenéis lo que debéis recordar, indispensablemente, para abordar con comodidad el curso de segundo de bachillerato. Aseguraos de que conocéis y recordáis todas estas formulas que aparecen en el `Kit de supervivencia para Matemáticas-2'.

\subsection{Preliminares}

El tema empieza con una breve introducción histórica a los problemas que dieron lugar al cálculo infinitesimal.

Se comenta la terminología que se va a utilizar: axiomas, teoremas, condiciones necesarias y suficientes, ...

Puesto que el cálculo se basa, entre otros conceptos, en el cuerpo de los números reales, se exponen una serie de propiedades de los mismos. Lo más importante del tema será tener muy claro todo lo relativo a las desigualdades y valor absoluto. Para ello se dispone de muchos ejercicios resueltos y propuestos al final del capítulo.

Acabamos con la importancia de las demostraciones en matemáticas con ejemplos del principio de inducción, la reducción al absurdo y el principio del palomar.

\subsection{Funciones}

Empezamos el tema con el concepto de función real de variable real, dominio, recorrido y gráfica.

Pasamos a estudiar las operaciones básicas con funciones así como la composición y la existencia de función inversa en casos de inyectividad.

Seguidamente se estudian las transformaciones básicas de funciones y se habla de las funciones definidas a trozos.

Acaba el tema con una colección de ejercicios (resueltos y propuestos, con solución) para afianzar los conceptos estudiados.

\subsection{Límites y Continuidad}

Después de introducir el concepto de continuidad con un ejemplo de la física pasamos a un estudio intuitivo de los límites así como una colección de ejercicios para repasar el cálculo de límites.  Posteriormente damos las definiciones formales de estos importantes conceptos.

Continuamos el tema con el estudio de la continuidad de las funciones y con otra colección de ejercicios sobre este concepto.

El tema acaba con los importantes teoremas sobre funciones continuas en intervalos cerrados: Bolzano, Darboux y Weiertrass. A estos teoremas también les acompañan una batería de ejercicios resueltos y propuestos.

\subsection{La Derivada}

En el tema se repasan el concepto de derivada y las reglas de derivación de las funciones elementales, así como la regla de la cadena. Todo ello se supone conocido de cursos anteriores.

Se incide en la relación entre continuidad y derivabilidad y se introducen nuevas técnicas de derivación: la derivación implícita, logarítmica y de la función inversa. Se aprende a derivar funciones a trozos y se introducen las derivadas enésimas. Acabamos el tema con el importante concepto de `diferencial' como primera aproximación (lineal) de una función.

Como siempre, acompañamos el tema con una buena colección de ejercicios resueltos paso a paso así como una colección de ejercicios propuestos con su solución.

\subsection{Aplicaciones de la derivada}

Rectas tangente y Normal a una función en un punto.

Información de la primera derivada (crecimiento y extremos, extremos absolutos, `optimización de funciones').

Información de la segunda derivada (curvatura e inflexiones). 

Problemas de encontar parámetros dadas determinadas condiciones de la función.

Teoremas relativos a funciones derivables.:  los teoremas vistos, el th. de Roll (con sus tres hipótesis) asegura cuando una función tiene un extremo en un intervalo. El Teorema del Valor Medio (con sus dos hipótesis) asegura que se puede, al menos en un punto, trazar la tangente paralela a la cuerda que une los extremos de la función. Importantísima la regla de L'Hôpital, se aplica límites de cocientes con indeterminación $0/0$ ó $\infty / \infty$.

Acabamos el tema con la representación gráfica de funciones explícitas.

Aparte de incluir ejemplos de muchos casos, también hemos introducido problemas resueltos y propuestos con solución de cada uno de los apartados anteriores.

Acabamos el tema, como siempre, con una colección de problemas resueltos y propuestos del tema.

\subsection{Desarrollos de Taylor y MacLaurin}

El tema, como el símbolo $\divideontimes$ indica, excede de los objetivos de un curso de segundo de bachillerato, pero se ha intentado presentar de forma sencilla para que sirva como introducción a aquellos/as alumnos/as que deseen ampliar sus conocimientos matemáticos de cara a afrontar, más preparados, la nueva carrera elegida. 

Unas sencillas consideraciones sobre espacios vectoriales nos permiten introducir la fórmula de Taylor para funciones polinómicas. Hemos hecho una generalización para aproximar, bajo determinadas condiciones, una función $f(x)$ por un polinomio de Taylor de orden $n, \; P_n(x)$ y, a la vez, dar una cota del error cometido al hacer tal aproximación a través del resto de Lagrange.

Se ven los desarrollos de MacLaurin y Taylor para alguna funciones conocidas y se muestran unos ejemplos de su potenciabilidad para cálculo de valores en funciones complicadas sin más que calcular el valor numérico de un polinomio.

Entre las aplicaciones que hemos visto, con ejemplos incluidos, de estos desarrollos están las Condición Necesaria y Suficiente  para que un punto sea Extremo Relativo o Punto de Inflexión. También hemos visto la aplicación de los desarrollos de Taylor y MacLaurin al cálculo de límites.

Debido a la dificultad del tema, solo contiene ejemplos y algún ejercicio resuelto, pero no ejercicios propuestos.

\subsection{La integral indefinida}

Prestad mucha atención a este tema, es importantísimo para alumnos/as que deseen cursar estudios superiores que necesiten de una ampliación matemática fuerte. \emph{`A la facultad se debe llegar sabiendo integrar'}.

En el tema se define el concepto de primitiva o anti-derivada de una función y su generalización a la integral indefinida, así como sus propiedades (linealidad de la integral indefinida).

Se encuentran las integrales de las funciones elementales y se extienden a las funciones compuestas (regla de la cadena), lo que da lugar a la `tabla de integrales inmediatas', que aparecerá más desarrollada en el apéndice \ref{app:tabla-integrales}.

A continuación se estudian algunos métodos de integración: integración por partes, integración de funciones racionales e integración por cambio de variable o sustitución, apartado en el que se ven algunos casos sencillos.

Se da, como ampliación, el `método de Hermite' de integración de funciones racionales con raíces complejas múltiples en el denominador.  

Se estudian, con ejemplos y ejercicios resueltos,  el cálculo de primitivas bajo determinadas condiciones.

Aunque el tema, como todos los demás, está repleto de ejemplos, ejercicios resueltos y ejercicios propuestos con solución para cada apartado, al final se ofrece una colección de unos 100 ejercicios propuestos con solución para practicar. Recordad: \emph{`A la facultad se debe llegar sabiendo integrar'}.

Acabamos el tema con un apartado de `curiosidades'.

\subsection{La integral definida}

Empezamos el tema con una introducción histórica al a la integral definida como la solución al problema de encontrar el área bajo una curva. Definimos qué se entiende por trapecios mixtilíneos, particiones del intervalo de integración y dos aproximaciones, una por exceso ($U$) y otra por defecto ($L$) de las sumas de Riemann. Haciendo un paso al límite y teniendo en cuenta el teorema del  criterio del Sandwich en el cálculo de límites que se ve en el tema 3, se define la integral definida como `suma de infinitas cantidades infinitesimales'.

Pasamos a ver que cualquier función continua en un intervalo $]a,b[$, o que tenga un número finito de discontinuidades evitables o de salto finito en el intervalo de integración es integrable. Seguidamente, con la idea en mente de que la ID (Integral Definida) representa el área encerrada por $f$ (si $f>0$), exponemos las `propiedades de la ID', para acabar viendo el `teorema de la media para el calculo integral', que permite calcular el valor medio de una función ctna. en un intervalo $<f>$.

En la segunda sección es donde relacionamos este nuevo concepto de ID -- integral definida--  (cálculo de áreas bajo curvas) con la noción de Derivada (cálculo de la pendiente de la recta tangente a una función en un punto). Definiendo una nueva función llamada `función área' y derivándola encontramos la sorprendente relación entre ambos conceptos: `Teorema Fundamental del Cálculo Integral y Regla de Barrow'. Unos corolarios a estos teoremas nos permiten derivar integrales.

Estudiamos el caso del cambio de variable en ID, que exige que éste sea inyectivo y permite extender el cambio elegido a los límites de integración.

Hablamos también de lo que llamaremos integrales impropias: cuando alguno (o ambos) de los límites de integración son $\infty$ y cuando $f$ presenta discontinuidades asintóticas (de salto infinito) en el intervalo de integración.

Terminamos esta sección con unos cuantos ejemplos de todo lo visto en ella.

En la tercera sección, aplicaciones de la ID, vemos las aplicaciones al cálculo de límites, \emph{el cálculo de áreas de figuras planas} (este es el más importante en segundo de bachillerato), el cálculo de longitudes de arcos de curvas, el cálculo de volúmenes de los cuerpos de revolución y de sus áreas.

De todo ellos vemos muchos ejemplos así como ejercicios resueltos y propuestos (con solución).

Acabamos (casi) el tema con una vasta colección de ejercicios resueltos y propuestos (con solución).

Para finalizar (ahora sí), presentamos una sección de `Curiosidades y Aplicaciones de la ID'.

\subsection{Introducción a las ecuaciones diferenciales}

El estudio de ecuaciones diferenciales (ecuaciones que involucran a la variable independiente $x$, a una función de ella $y=f(x)$ y a sus derivadas $y', \; y'', ; \cdots$)  es un campo extenso en matemáticas puras y aplicadas, en física y en la ingeniería. Todas estas disciplinas se interesan en las propiedades de ecuaciones diferenciales de varios tipos.

Sin embargo, el contenido de este tema excede al nivel de bachillerato.  Es copia `casi' textual de los apuntes del catedrático J.A. Gálvez, del departamento de Geometría y Topología de la Universidad de Granada, que encontré en internet y al cual agradezco desde aquí su trabajo docente expuesto al público en general.

Si el lector/a lo considera demasiado complicado, puede saltar al siguiente capítulo, pero su importancia es crucial en física e ingenierías.

\subsection{Introducción al cálculo vectorial en $\mathbb R^3$}

En este tema se introducen los campos escalares y vectoriales, así como los conceptos de derivadas parciales, gradiente, operador nabla, derivada direccional, divergencia, rotacional, laplaciana y operador laplacina. Si al lector le resulta muy tediosso, puede pasarlo por alto, pero su importancia es crucial en física e ingenierías.

\subsection{Apéndices}
\begin{enumerate}[A) ]
\item Kit de supervivencia para matemáticas-ii
	\begin{itemize}
	\item Aritmética y Álgebra
	\item Complejos
	\item Trigonometría
	\item Geometría analítica y vectores 2D	
	\item Análisis
	\end{itemize}
\item Gráficas de las funciones elementales más conocidas
\item Tabla de derivadas
\item Desarrollos es serie de MacLaurin más usuales
\item Tabla de integrales. 
\end{enumerate}


\begin{figure}[H]
	\centering
	\includegraphics[width=1\textwidth]{imagenes/imagenes00/xiste00.png}
\end{figure}


	\textit{Este material es un conjunto de apuntes personales, que comparto gratuitamente en la red, basados en mi experiencia como profesor, varios textos citados anteriormente y webs de internet. Si hay algún contenido que no he incluido correctamente, hacédmelo saber por e-mail y lo editaré como se pida.  También se agradecería la comunicación de la detección de cualquier error.}

\vspace{10mm}
\emph{Este documento se comparte bajo licencia `Attribution-NonCommercial 4.0 International (CC BY-NC 4.0)'}

\vspace{10mm}

\begin{multicols}{2}
\begin{figure}[H]
	\centering
	\includegraphics[width=.4
	\textwidth]{imagenes/imagenes00/licencia.png}
\end{figure}
\begin{figure}[H]
	\centering
	\includegraphics[width=.3
	\textwidth]{imagenes/firma.png}
\end{figure}
\end{multicols}


